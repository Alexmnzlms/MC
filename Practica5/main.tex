\documentclass[12pt, spanish]{article}
\usepackage[spanish]{babel}
\selectlanguage{spanish}
\usepackage{url}
\usepackage[utf8x]{inputenc}
\usepackage{graphicx}
\graphicspath{{images/}}
\usepackage{parskip}
\usepackage{fancyhdr}
\usepackage{vmargin}
\usepackage{listings}
\usepackage{tikz}

\usepackage[default]{sourcesanspro}

\setmarginsrb{2 cm}{1 cm}{2 cm}{2 cm}{1 cm}{1.5 cm}{1 cm}{1.5 cm}

\title{Práctica 5}
\author{Alejandro Manzanares Lemus }
\date{\today}

\makeatletter
\let\thetitle\@title
\let\theauthor\@author
\let\thedate\@date
\makeatother

\pagestyle{fancy}
\fancyhf{}
\rhead{\theauthor}
\lhead{\thetitle}
\cfoot{\thepage}

\begin{document}

\begin{titlepage}
    \centering
    \vspace*{0.5 cm}
    \includegraphics[scale = 0.50]{UGR.png}\\[1.0 cm]
    %\textsc{\LARGE Universidad de Granada}\\[2.0 cm]
    \textsc{\large 3ºA}\\[0.5 cm]
    \textsc{\large Grado en Ingeniería Informática}\\[0.5 cm]
    \rule{\linewidth}{0.2 mm} \\[0.4 cm]
    { \huge \bfseries \thetitle}\\
    \rule{\linewidth}{0.2 mm} \\[1.5 cm]

    \begin{minipage}{0.4\textwidth}
        \begin{flushleft} \large
            \emph{Autor:}\\
            \theauthor
            \end{flushleft}
            \end{minipage}~
            \begin{minipage}{0.4\textwidth}
            \begin{flushright} \large
            \emph{Asignatura: \\
            Modelos de Computación}
        \end{flushright}
    \end{minipage}\\[1 cm]

    {\large \thedate}\\[1 cm]

    \vfill

\end{titlepage}
\pagebreak
\section{Ejercicio 1}
\subsection{Apartado a}
\begin{math}
S\rightarrow AbB\\
A\rightarrow aA | \epsilon\\
B\rightarrow aB|bB|\epsilon
\end{math}

La gramática de este apartado genera el lenguaje \begin{math}
L = \{a^{*}b(a,b)^{*}\}
\end{math}
, nunca podremos llegar a generar la misma palabra a través de dos caminos, porque partiendo de la producción inicial  \begin{math} S \rightarrow AbB \end{math}, siguiendo por A solo podremos añadir a's a la izquierda de la palabra y siguiendo por B solo podemos añadir a's o b's al final de la palabra.

\subsection{Apartado b}
\begin{math}
S \rightarrow abaS|babS|baS|\epsilon
\end{math}

Esta gramática es ambigua, porque existe al menos una palabra que se puede generar dos veces siguiendo dos caminos distintos.\\
La palabra bababa se puede formar de las siguientes formas:\\
\begin{math}
S \rightarrow babS \rightarrow bababaS \rightarrow bababa\\
S \rightarrow baS \rightarrow babaS \rightarrow bababaS \rightarrow bababa
\end{math}

\section{Ejercicio 2}

\begin{math}
L = \{0^{i}1^{j}2^{k}3^{m}4 \end{math} tal que \begin{math} i,j,k \ge, m = i+j+k\}
\end{math}

\begin{right}
\begin{tikzpicture}[scale=0.2]
\tikzstyle{every node}+=[inner sep=0pt]
\draw [black] (6.1,-3.2) circle (3);
\draw (6.1,-3.2) node {$q0$};
\draw [black] (32.5,-3.2) circle (3);
\draw (32.5,-3.2) node {$q1$};
\draw (6.1,-13.8) node {$0/C/CC$};
\draw (6.1,-16) node {$1/Z/UZ$};
\draw (6.1,-18.2) node {$1/C/UC$};
\draw (6.1,-20.6) node {$1/U/UU$};
\draw (6.1,-22.8) node {$2/Z/DZ$};
\draw (6.1,-25) node {$2/C/DC$};
\draw (6.1,-27.4) node {$2/U/DU$};
\draw (6.1,-29.5) node {$2/D/DD$};
\draw (19.4,-8.9) node {$\epsilon/U/U$};
\draw (19.4,-11.2) node {$\epsilon/D/D$};
\draw (19.4,-6.8) node {$\epsilon/C/C$};
\draw (32.5,-13.8) node {$3/U/\epsilon$};
\draw (32.5,-16) node {$3/D/\epsilon$};
\draw (32.5,-18.2) node {$4/Z/\epsilon$};
\draw [black] (9.1,-3.2) -- (29.5,-3.2);
\fill [black] (29.5,-3.2) -- (28.7,-2.7) -- (28.7,-3.7);
\draw (19.3,-3.7) node [below] {$\epsilon/Z/Z$};
\draw [black] (33.823,-5.88) arc (54:-234:2.25);
\draw (32.5,-10.45) node [below] {$3/C/\epsilon$};
\fill [black] (31.18,-5.88) -- (30.3,-6.23) -- (31.11,-6.82);
\draw [black] (7.423,-5.88) arc (54:-234:2.25);
\draw (6.1,-10.45) node [below] {$0/Z/CZ$};
\fill [black] (4.78,-5.88) -- (3.9,-6.23) -- (4.71,-6.82);
\draw [black] (0.2,-3.2) -- (3.1,-3.2);
\fill [black] (3.1,-3.2) -- (2.3,-2.7) -- (2.3,-3.7);
\end{tikzpicture}
\end{right}

La pila empieza con una Z.\\
Cada vez que el autómata lee un 0 desplaza el tope de la pila y mete una C, cada vez que lee un 1 una U y cada vez que lee un 2 una D. \\
Como 0, 1 y 2 deben permanecer en ese orden, cuando se lea un cero solo se meterá una C en la pila si el tope de esta fuera una Z (no se ha introducido ningún cero) o una C (ya se ha introducido algún cero), cuando se lea un 1 solo se meterá una U si el tope de la pila es Z, C o U (ya se ha introducido algún uno) y como es de esperar, cuando se lea un 2, solo se meterá una D en la pila si el tope es Z, C, U o D (ya se ha introducido algún dos).\\
Una vez hemos terminado de introducir 0's, 1's y 2's, por una transición nula pasamos a otro estado, en el que el autómata cada vez que lea un 3, si el tope de la pila es C, U o D lo saca de esta, porque el numero de 3's debe ser la suma del numero de 0's, 1's y 2's. Si no hubiéramos introducido ninguno, el numero de 3 seria 0.\\
Cuando el autómata lea un 4, se supone que el único elemento que debe quedar en la pila es el símbolo inicial, y lo que hace es sacarlo de la pila y se acepta la palabra.
\section{Ejercicio 3}
\begin{math}
L = \{1^{n}w \in \{0,1\} \end{math} tal que \begin{math} |w| =n,n > 0\}
\end{math}
\subsection{Autómata}
\begin{left}
\begin{tikzpicture}[scale=0.2]
\tikzstyle{every node}+=[inner sep=0pt]
\draw [black] (6.6,-3.2) circle (3);
\draw (6.6,-3.2) node {$q0$};
\draw [black] (39.6,-3.2) circle (3);
\draw (39.6,-3.2) node {$q1$};
\draw (6.6,-13.4) node {$1/S/SS$};
\draw (39.6,-13.4) node {$1/S/\epsilon$};
\draw (39.6,-15.2) node {$\epsilon/Z/\epsilon$};
\draw [black] (7.923,-5.88) arc (54:-234:2.25);
\draw (6.6,-10.45) node [below] {$1/Z/SZ$};
\fill [black] (5.28,-5.88) -- (4.4,-6.23) -- (5.21,-6.82);
\draw [black] (9.6,-3.2) -- (36.6,-3.2);
\fill [black] (36.6,-3.2) -- (35.8,-2.7) -- (35.8,-3.7);
\draw (23.1,-3.7) node [below] {$\epsilon/S/S$};
\draw [black] (40.923,-5.88) arc (54:-234:2.25);
\draw (39.6,-10.45) node [below] {$0/S/\epsilon$};
\fill [black] (38.28,-5.88) -- (37.4,-6.23) -- (38.21,-6.82);
\draw [black] (0.2,-3.2) -- (3.6,-3.2);
\fill [black] (3.6,-3.2) -- (2.8,-2.7) -- (2.8,-3.7);
\end{tikzpicture}
\end{left}
\subsection{Gramática}
\begin{math}
S \rightarrow (q_{0},Z,q_{0})|(q_{0},Z,q_{1})\\
(q_{0},Z,q_{0}) \rightarrow 1(q_{0},S,q_{0})(q_{0},Z,q_{0}) | 1(q_{0},S,q_{1})(q_{1},Z,q_{0})\\
(q_{0},Z,q_{1}) \rightarrow 1(q_{0},S,q_{0})(q_{0},Z,q_{1}) | 1(q_{0},S,q_{1})(q_{1},Z,q_{1})\\
(q_{1},Z,q_{1}) \rightarrow \epsilon\\
(q_{0},S,q_{0}) \rightarrow 1(q_{0},S,q_{0})(q_{0},S,q_{0}) | 1(q_{0},S,q_{1})(q_{1},S,q_{0})\\
(q_{0},S,q_{1}) \rightarrow 1(q_{0},S,q_{0})(q_{0},S,q_{1}) | 1(q_{0},S,q_{1})(q_{1},S,q_{1}) | (q_{1},S,q_{1})\\
(q_{1},S,q_{1}) \rightarrow 0 | 1\\
\end{math}
\subsection{Eliminar producciones inútiles}
Renombremos las reglas:\\
\begin{math}
A = (q_{0},Z,q_{0})\\
B = (q_{0},Z,q_{1})\\
C = (q_{1},Z,q_{1})\\
D = (q_{0},S,q_{0})\\
E = (q_{0},S,q_{1})\\
F = (q_{1},S,q_{1}\\
\end{math}
Eliminamos \begin{math}
(q_{1},Z,q_{0}) y (q_{1},S,q_{0})
\end{math} porque a partir de ellas no podemos generar nada y nos queda:
S \rightarrow A|B\\
A \rightarrow 1DA | 1E\\
B \rightarrow 1DB | 1EC\\
C \rightarrow \epsilon\\
D \rightarrow 1DD | 1E\\
E \rightarrow 1DE | 1EF | F\\
F \rightarrow 0 | 1\\

Eliminamos las producciones inútiles:\\
S \rightarrow B\\
B \rightarrow 1E\\
E \rightarrow 1EF | F\\
F \rightarrow 0 | 1\\

\subsection{Forma normal de Chomsky}
Para pasar a forma normal de Chomsky simplemente tenemos que crear una nueva regla para generar EF y una para generar 1:\\
S \rightarrow B\\
B \rightarrow UE\\
E \rightarrow UA | F\\
F \rightarrow 0 | 1\\
A \rightarrow EF\\
U \rightarrow 1

\end{document}
